%Version 3.1 December 2024
% See section 11 of the User Manual for version history
%
%%%%%%%%%%%%%%%%%%%%%%%%%%%%%%%%%%%%%%%%%%%%%%%%%%%%%%%%%%%%%%%%%%%%%%
%%                                                                 %%
%% Please do not use \input{...} to include other tex files.       %%
%% Submit your LaTeX manuscript as one .tex document.              %%
%%                                                                 %%
%% All additional figures and files should be attached             %%
%% separately and not embedded in the \TeX\ document itself.       %%
%%                                                                 %%
%%%%%%%%%%%%%%%%%%%%%%%%%%%%%%%%%%%%%%%%%%%%%%%%%%%%%%%%%%%%%%%%%%%%%

%%\documentclass[referee,sn-basic]{sn-jnl}% referee option is meant for double line spacing

%%=======================================================%%
%% to print line numbers in the margin use lineno option %%
%%=======================================================%%

%%\documentclass[lineno,pdflatex,sn-basic]{sn-jnl}% Basic Springer Nature Reference Style/Chemistry Reference Style

%%=========================================================================================%%
%% the documentclass is set to pdflatex as default. You can delete it if not appropriate.  %%
%%=========================================================================================%%

%%\documentclass[sn-basic]{sn-jnl}% Basic Springer Nature Reference Style/Chemistry Reference Style

%%Note: the following reference styles support Namedate and Numbered referencing. By default the style follows the most common style. To switch between the options you can add or remove “Numbered” in the optional parenthesis. 
%%The option is available for: sn-basic.bst, sn-chicago.bst%  
 
%%\documentclass[pdflatex,sn-nature]{sn-jnl}% Style for submissions to Nature Portfolio journals
%%\documentclass[pdflatex,sn-basic]{sn-jnl}% Basic Springer Nature Reference Style/Chemistry Reference Style
\documentclass[pdflatex,sn-mathphys-num]{sn-jnl}% Math and Physical Sciences Numbered Reference Style
%%\documentclass[pdflatex,sn-mathphys-ay]{sn-jnl}% Math and Physical Sciences Author Year Reference Style
%%\documentclass[pdflatex,sn-aps]{sn-jnl}% American Physical Society (APS) Reference Style
%%\documentclass[pdflatex,sn-vancouver-num]{sn-jnl}% Vancouver Numbered Reference Style
%%\documentclass[pdflatex,sn-vancouver-ay]{sn-jnl}% Vancouver Author Year Reference Style
%%\documentclass[pdflatex,sn-apa]{sn-jnl}% APA Reference Style
%%\documentclass[pdflatex,sn-chicago]{sn-jnl}% Chicago-based Humanities Reference Style

%%%% Standard Packages
%%<additional latex packages if required can be included here>

\usepackage{graphicx}%
\usepackage{multirow}%
\usepackage{amsmath,amssymb,amsfonts}%
\usepackage{amsthm}%
\usepackage{mathrsfs}%
\usepackage[title]{appendix}%
\usepackage{xcolor}%
\usepackage{textcomp}%
\usepackage{manyfoot}%
\usepackage{booktabs}%
\usepackage{algorithm}%
\usepackage{algorithmicx}%
\usepackage{algpseudocode}%
\usepackage{listings}%

\usepackage{kotex}
%%%%

%%%%%=============================================================================%%%%
%%%%  Remarks: This template is provided to aid authors with the preparation
%%%%  of original research articles intended for submission to journals published 
%%%%  by Springer Nature. The guidance has been prepared in partnership with 
%%%%  production teams to conform to Springer Nature technical requirements. 
%%%%  Editorial and presentation requirements differ among journal portfolios and 
%%%%  research disciplines. You may find sections in this template are irrelevant 
%%%%  to your work and are empowered to omit any such section if allowed by the 
%%%%  journal you intend to submit to. The submission guidelines and policies 
%%%%  of the journal take precedence. A detailed User Manual is available in the 
%%%%  template package for technical guidance.
%%%%%=============================================================================%%%%

%% as per the requirement new theorem styles can be included as shown below
\theoremstyle{thmstyleone}%
\newtheorem{theorem}{Theorem}%  meant for continuous numbers
%%\newtheorem{theorem}{Theorem}[section]% meant for sectionwise numbers
%% optional argument [theorem] produces theorem numbering sequence instead of independent numbers for Proposition
\newtheorem{proposition}[theorem]{Proposition}% 
%%\newtheorem{proposition}{Proposition}% to get separate numbers for theorem and proposition etc.

\theoremstyle{thmstyletwo}%
\newtheorem{example}{Example}%
\newtheorem{remark}{Remark}%

\theoremstyle{thmstylethree}%
\newtheorem{definition}{Definition}%

\raggedbottom
%%\unnumbered% uncomment this for unnumbered level heads

\begin{document}

\title[Article Title]{A Prototype of Live Mobile Learning System Using RTMP and STOMP}

%%=============================================================%%
%% GivenName	-> \fnm{Joergen W.}
%% Particle	-> \spfx{van der} -> surname prefix
%% FamilyName	-> \sur{Ploeg}
%% Suffix	-> \sfx{IV}
%% \author*[1,2]{\fnm{Joergen W.} \spfx{van der} \sur{Ploeg} 
%%  \sfx{IV}}\email{iauthor@gmail.com}
%%=============================================================%%

\author{\fnm{Hosung} \sur{Kim}}\email{hosungkim5108@gmail.com}

%%\author[2,3]{\fnm{Second} \sur{Author}}\email{iiauthor@gmail.com}
%%\equalcont{These authors contributed equally to this work.}

%%\author[1,2]{\fnm{Third} \sur{Author}}\email{iiiauthor@gmail.com}
%%\equalcont{These authors contributed equally to this work.}

\affil{\orgdiv{Department of Computer Engineering}, \orgname{Hongik University}, \orgaddress{ \city{Seoul}, \country{Korea}}}

%%\affil[2]{\orgdiv{Department}, \orgname{Organization}, \orgaddress{\street{Street}, \city{City}, \postcode{10587}, \state{State}, \country{Country}}}

%%\affil[3]{\orgdiv{Department}, \orgname{Organization}, \orgaddress{\street{Street}, \city{City}, \postcode{610101}, \state{State}, \country{Country}}}

%%==================================%%
%% Sample for unstructured abstract %%
%%==================================%%

\abstract{본 보고서는 실시간 상호작용이 가능한 모바일 원격 교육 시스템의 프로토타입을 설계 및 구현한 내용을 다룬다. 제안된 시스템은 iPad 기반 환경에서 동작하며, 강사의 음성과 전면 카메라 영상, PDF 강의자료 및 주석을 하나의 스트림으로 합성하여 RTMP 프로토콜을 통해 송출한다. 또한, STOMP 기반 WebSocket 채팅 기능을 통해 강사와 학생 간의 텍스트 기반 실시간 소통을 지원하며, PDFKit을 활용한 주석 기능과 ReplayKit을 통한 화면 송출도 포함된다. 실제 테스트 환경에서는 강사 1인, 학생 2인을 대상으로 1초 이내의 지연으로 원활한 실시간 강의와 상호작용이 가능함을 확인하였다. 향후에는 다중 강의실 지원, 참여자 현황 관리, 강사 카메라 프리뷰, 화이트보드 기능 등을 추가하여 시스템의 확장성과 완성도를 높이고자 한다.}

%%\keywords{keyword1, Keyword2, Keyword3, Keyword4}

%%\pacs[JEL Classification]{D8, H51}

%%\pacs[MSC Classification]{35A01, 65L10, 65L12, 65L20, 65L70}

\maketitle

\section{Introduction}\label{sec1}

COVID-19 팬데믹의 영향으로 온라인 기반 원격 교육의 수요가 급증하면서, 원격 교육의 중요성을 부각시키는 결정적인 계기가 되었다. 팬데믹 이후에도 원격 교육은 공간적 제약 없는 학습 환경에 대한 지속적인 수요를 충족시키며, 위기 상황의 일시적인 대안을 넘어 새로운 교육 패러다임으로 자리매김하였다.

특히, 스마트폰과 태블릿 등의 모바일 디지털 기기의 보급 확대는 모바일 원격 교육 시스템의 가능성을 열어주었다. 모바일 원격 교육 시스템은 강사와 학생 간의 즉각적인 피드백을 가능하게 하며 실시간 상호작용을 통해 학습 효과를 극대화한다.

이에 본 보고서에서는 모바일 원격 교육 시스템에 대한 이해를 목적으로 텍스트 기반 채팅, 강사의 음성 및 영상 공유, PDF 기반 강의 자료 주석 기능 등을 포함한 데모 애플리케이션을 설계 및 구현하였다. 제시된 시스템은, 강사가 iPad로 강의를 진행하며, 학생들은 iPad를 이용하여 강사의 음성과 전면 카메라 영상, PDF 강의 자료 및 애노테이션을 실시간으로 보고 들으면서 상호작용을 할 수 있는 형태로 되어 있다.

\section{Design}\label{sec2}

\subsection{Structure}\label{subsec1}

본 시스템은 강사측 모바일 클라이언트, 학생측 모바일 클라이언트, 스트리밍 서버, 채팅 서버로 구성된다. 전체 시스템 구성은 그림\ref{fig1}과 같다.

\begin{figure}[h]
\centering
\includegraphics[width=0.9\textwidth]{architecture_of_system.png}
\caption{Structure of a system}\label{fig1}
\end{figure}

\subsection{Streaming}\label{subsec2}

강사측 모바일 클라이언트는 음성과 영상을 실시간으로 스트리밍 서버에 전송하고, 학생측 모바일 클라이언트는 스트리밍 서버로부터 데이터를 수신한다. 강사측 모바일 클라이언트에서 전송하는 영상은 강사의 iPad 화면에 표시된 PDF자료와 주석을 녹화한 화면과 전면카메라를 하나의 스트림으로 합성하여 송출한다.

\subsection{Chat}\label{subsec3}

강사와 학생의 모바일 클라이언트는 채팅 서버와 각각 연결되며, 서버는 수신한 메세지를 모든 연결된 세션에게 브로드캐스트한다. 이를 통해 텍스트 기반의 양방향 소통이 가능하며, 강의 중 질문이나 피드백을 실시간으로 주고받을 수 있다.

\subsection{Architecture}\label{subsec4}

\begin{figure}[h]
\centering
\includegraphics[width=0.9\textwidth]{architecture_of_client.png}
\caption{Architecture of a client}\label{fig2}
\end{figure}

클라이언트 애플리케이션은 Clean Architecture와 MVVM(Model - View - ViewModel) 아키텍쳐 패턴을 결합하여 설계되었다. Clean Architecture는 애플리케이션을 계층화하여 각 계층의 책임을 명확하게 함으로써, 재사용성과 테스트 용이성을 높이는 데 목적이 있다. 본 애플리케이션은 그림\ref{fig2}에서 볼 수 있듯이 Data, Domain, Presentation 세 계층으로 구성되고 각 계층에 대한 자세한 설명은 다음과 같다.

Data 계층은 Network, DTO, Repository(Implementation)로 구성되며, 서버와의 직접적인 통신을 담당한다. Domain 계층은 Entity, Repository(Protocol), UseCase로 구성되며, 핵심 데이터와 비즈니스 로직을 포함한다. Presentation 계층은 ViewModel, ViewController로 구성되고, UI를 통해 User와 직접 소통한다.

서버와 클라이언트 간의 데이터 흐름은 다음과 같다.
서버로부터 데이터를 수신하는 경우, Network가 서버로부터 DTO 형태의 데이터를 전달받는다. 해당 데이터는 Repository에서 Entity로 변환되고 UseCase를 거쳐 ViewModel에 전달된다. ViewController는 ViewModel의 상태 변화를 관찰하고 있다가 변경사항이 발생하면 이를 UI에 반영한다.
서버로 데이터를 전송하는 경우, ViewController에서 ViewModel, UseCase, Repository를 통해 값을 전달한다. Repository를 거치며 Entity는 DTO형태로 변환되고 최종적으로 Network를 통해 서버로 전송한다.

\section{Implementation}\label{sec3}

\begin{figure}[p]
\centering
\includegraphics[width=0.9\textwidth]{intructor.jpeg}
\caption{User interface of the instructor's client}\label{fig3}
\end{figure}

\begin{figure}[p]
\centering
\includegraphics[width=0.9\textwidth]{student.jpeg}
\caption{User interface of the student's client}\label{fig4}
\end{figure}

\subsection{Streaming}\label{subsec5}

스트리밍 기능은 RTMP(Real-Time Messaging Protocol)\cite{RTMP}를 기반으로 구현하였다. 초기 설계 단계에서는 WebRTC 프로토콜의 도입도 고려되었으나, 본 데모 애플리케이션의 구조상 Instructor 클라이언트는 송출 전용, Student 클라이언트는 수신전용의 단방향 흐름을 갖기 때문에, 구현의 단순성과 안정성을 고려하여 RTMP를 최종적으로 채택하였다.

서버는 NGINX\cite{NGINX}기반의 RTMP 스트리밍 서버를 사용하였으며, nginx.conf 설정 파일만 수정함으로써 간단하게 구축할 수 있다.

클라이언트에서는 HaishinKit.swift\cite{HaishinKit} 라이브러리를 활용하여 RTMP 프로토콜을 통한 미디어 송수신을 담당한다. \verb+RTMPService+ 클래스는 해당 라이브러리를 통해 RTMP 서버와의 연결을 설정하고 Instructor 또는 Student의 역할에 따라 수신이나 송출의 역할을 한다. \verb+StreamService+ 클래스는 HaishinKit.swift 라이브러리의 \verb+MediaMixer+를 활용하여 강사의 음성, 강사의 전면 카메라 영상, 화면 녹화(PDF 및 주석)를 하나의 스트림으로 합성한다.

\subsection{Chat}\label{subsec6}

채팅은 STOMP(Simple Text Oriented Messaging Protocol)\cite{STOMP} 기반의 Websocket 통신으로 구현되었다. 서버는 Spring Boot Starter Web Socket Framework를 사용하여 개발하였으며, 클라이언트는 SwiftStomp\cite{SwiftStomp} 라이브러리를 사용하여 개발하였다.

SwiftStomp는 STOMP 메세지 전송 및 구독, 핑-퐁 핸들링, 자동 재연결 등의 기능을 지원하므로 안정적인 통신 환경을 제공한다. \verb+WebSocketService+는 SwiftStomp 라이브러리를 사용하여 서버와 연결을 설정하고 메세지 송수신을 담당한다. SwiftStomp의 \verb+eventsUpStream+, \verb+messagesUpstream+, \verb+receiptUpstream+ 스트림들을 Combine 프레임워크의 \verb+PassthroughSubject+와 연동하여 해당 메세지를 방출하고 ViewController에서 이를 구독하고있다가 UI에 반영한다.

\subsection{PDF and Annotation}\label{subsec7}

강의 자료 공유는 PDF 문서를 기반으로 이루어지며 Apple 기본 프레임워크 PDFKit\cite{PDFKit}을 활용하여 구현하였다. \verb+UIDocumentPickerViewController+를 통해 기기에 저장된 PDF 파일을 선택하면, 해당 파일의 링크를 가져와서 \verb+PDFView+로 화면에 출력한다.

애노테이션은 PDFKit의 \verb+PDFPageOverlayViewProvider+를 사용하여 각 \verb+PDFPage+에 \verb+CanvasView+를 \verb+overlayView+로 추가하여 구현하였다. \verb+CanvasView+는 터치 이벤트를 \verb+override+하여 \verb+UIBezierPath+로 배열에 저장하는 방식으로 동작한다.

PDF 문서 스크롤과 그리기 기능 간의 입력 충돌을 해결하기 위해, \verb+PDFView+의 \verb+isScrollEnabled+ 속성을 제어하는 그리기 모드 토글 버튼을 구현하였다. 이 버튼을 통해 사용자는 스크롤과 주석 작성 기능을 전환할 수 있다.

강사의 화면은 ReplayKit\cite{ReplayKit}을 통해 녹화되어, 음성 및 전면 카메라 영상과 함께 송출되며, 이 과정에서 \verb+PDFView+와 애노테이션 역시 함께 전달된다.

\section{Conclusion}\label{sec4}

\begin{figure}[h]
\centering
\includegraphics[width=0.9\textwidth]{test.jpeg}
\caption{Test}\label{fig3}
\end{figure}

본 보고서에서는 iPadOS 기반의 모바일 원격 교육 시스템에 대한 이해를 목적으로 데모 애플리케이션을 설계 및 구현하였다.

MacBook M3 Pro에서 스트리밍 서버와 채팅 서버를 구동하고, iPad Air 3대를 활용하여 강사 1인과 학생2인의 환경에서 테스트를 진행한 결과, 1초 이내의 지연으로 실시간 강의 및 상호작용이 원할하게 수행됨을 확인하였다.

다만, 현재까지 개발된 프로토타입 시스템에서는 단일 강의실 환경에서 제한된 인원으로만 테스트된 초기 단계의 시스템으로, 실제 서비스 환경에 적용되기 위해서는 보다 다양한 환경에서의 테스트와 기능 확장이 요구된다.

따라서 향후에는 다음과 같은 기능을 추가하여 시스템의 완성도를 높이고자 한다.
첫째, 현재와 같은 단일 강의실 환경이 아닌, 사용자가 강의실을 생성하고 선택적으로 입장할 수 있는 기능을 추가하여 다수의 강의실이 동시에 운영될 수 있는 구조로 개선할 것이다.
둘째, 강의자가 실시간으로 참여 중인 학생 명단을 확인할 수 있는 기능을 추가하여 출석 관리 및 학습자 모니터링을 용이하게 할 것이다.
셋째, 강의자가 자신의 카메라 화면을 확인할 수 있는 기능을 추가하여 강의 진행의 안정성을 높일 것이다.
넷째, 화이트보드 기능을 추가하여 교육의 질을 높일 것이다.

전체 소스코드는 \url{https://github.com/H0sungKim/CollaborativeComputingLab}에서 확인할 수 있다.

%===========================================================================================%%
%% If you are submitting to one of the Nature Portfolio journals, using the eJP submission   %%
%% system, please include the references within the manuscript file itself. You may do this  %%
%% by copying the reference list from your .bbl file, paste it into the main manuscript .tex %%
%% file, and delete the associated \verb+\bibliography+ commands.                            %%
%%===========================================================================================%%

\bibliography{bibliography}

\end{document}
